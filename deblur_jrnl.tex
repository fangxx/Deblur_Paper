
\documentclass[journal]{IEEEtran}

\usepackage{cite}

\ifCLASSINFOpdf
  % \usepackage[pdftex]{graphicx}
  % declare the path(s) where your graphic files are
  % \graphicspath{{../pdf/}{../jpeg/}}
  % and their extensions so you won't have to specify these with
  % every instance of \includegraphics
  % \DeclareGraphicsExtensions{.pdf,.jpeg,.png}
\else
  % or other class option (dvipsone, dvipdf, if not using dvips). graphicx
  % will default to the driver specified in the system graphics.cfg if no
  % driver is specified.
  % \usepackage[dvips]{graphicx}
  % declare the path(s) where your graphic files are
  % \graphicspath{{../eps/}}
  % and their extensions so you won't have to specify these with
  % every instance of \includegraphics
  % \DeclareGraphicsExtensions{.eps}
\fi

\usepackage{amsmath}


% *** SPECIALIZED LIST PACKAGES ***
%
%\usepackage{algorithmic}
% algorithmic.sty was written by Peter Williams and Rogerio Brito.
% This package provides an algorithmic environment fo describing algorithms.
% You can use the algorithmic environment in-text or within a figure
% environment to provide for a floating algorithm. Do NOT use the algorithm
% floating environment provided by algorithm.sty (by the same authors) or
% algorithm2e.sty (by Christophe Fiorio) as the IEEE does not use dedicated
% algorithm float types and packages that provide these will not provide
% correct IEEE style captions. The latest version and documentation of
% algorithmic.sty can be obtained at:
% http://www.ctan.org/pkg/algorithms
% Also of interest may be the (relatively newer and more customizable)
% algorithmicx.sty package by Szasz Janos:
% http://www.ctan.org/pkg/algorithmicx




% *** ALIGNMENT PACKAGES ***
%
%\usepackage{array}
% Frank Mittelbach's and David Carlisle's array.sty patches and improves
% the standard LaTeX2e array and tabular environments to provide better
% appearance and additional user controls. As the default LaTeX2e table
% generation code is lacking to the point of almost being broken with
% respect to the quality of the end results, all users are strongly
% advised to use an enhanced (at the very least that provided by array.sty)
% set of table tools. array.sty is already installed on most systems. The
% latest version and documentation can be obtained at:
% http://www.ctan.org/pkg/array


% IEEEtran contains the IEEEeqnarray family of commands that can be used to
% generate multiline equations as well as matrices, tables, etc., of high
% quality.




% *** SUBFIGURE PACKAGES ***
%\ifCLASSOPTIONcompsoc
%  \usepackage[caption=false,font=normalsize,labelfont=sf,textfont=sf]{subfig}
%\else
%  \usepackage[caption=false,font=footnotesize]{subfig}
%\fi
% subfig.sty, written by Steven Douglas Cochran, is the modern replacement
% for subfigure.sty, the latter of which is no longer maintained and is
% incompatible with some LaTeX packages including fixltx2e. However,
% subfig.sty requires and automatically loads Axel Sommerfeldt's caption.sty
% which will override IEEEtran.cls' handling of captions and this will result
% in non-IEEE style figure/table captions. To prevent this problem, be sure
% and invoke subfig.sty's "caption=false" package option (available since
% subfig.sty version 1.3, 2005/06/28) as this is will preserve IEEEtran.cls
% handling of captions.
% Note that the Computer Society format requires a larger sans serif font
% than the serif footnote size font used in traditional IEEE formatting
% and thus the need to invoke different subfig.sty package options depending
% on whether compsoc mode has been enabled.
%
% The latest version and documentation of subfig.sty can be obtained at:
% http://www.ctan.org/pkg/subfig




% *** FLOAT PACKAGES ***
%
%\usepackage{fixltx2e}
% fixltx2e, the successor to the earlier fix2col.sty, was written by
% Frank Mittelbach and David Carlisle. This package corrects a few problems
% in the LaTeX2e kernel, the most notable of which is that in current
% LaTeX2e releases, the ordering of single and double column floats is not
% guaranteed to be preserved. Thus, an unpatched LaTeX2e can allow a
% single column figure to be placed prior to an earlier double column
% figure.
% Be aware that LaTeX2e kernels dated 2015 and later have fixltx2e.sty's
% corrections already built into the system in which case a warning will
% be issued if an attempt is made to load fixltx2e.sty as it is no longer
% needed.
% The latest version and documentation can be found at:
% http://www.ctan.org/pkg/fixltx2e


%\usepackage{stfloats}
% stfloats.sty was written by Sigitas Tolusis. This package gives LaTeX2e
% the ability to do double column floats at the bottom of the page as well
% as the top. (e.g., "\begin{figure*}[!b]" is not normally possible in
% LaTeX2e). It also provides a command:
%\fnbelowfloat
% to enable the placement of footnotes below bottom floats (the standard
% LaTeX2e kernel puts them above bottom floats). This is an invasive package
% which rewrites many portions of the LaTeX2e float routines. It may not work
% with other packages that modify the LaTeX2e float routines. The latest
% version and documentation can be obtained at:
% http://www.ctan.org/pkg/stfloats
% Do not use the stfloats baselinefloat ability as the IEEE does not allow
% \baselineskip to stretch. Authors submitting work to the IEEE should note
% that the IEEE rarely uses double column equations and that authors should try
% to avoid such use. Do not be tempted to use the cuted.sty or midfloat.sty
% packages (also by Sigitas Tolusis) as the IEEE does not format its papers in
% such ways.
% Do not attempt to use stfloats with fixltx2e as they are incompatible.
% Instead, use Morten Hogholm'a dblfloatfix which combines the features
% of both fixltx2e and stfloats:
%
% \usepackage{dblfloatfix}
% The latest version can be found at:
% http://www.ctan.org/pkg/dblfloatfix




%\ifCLASSOPTIONcaptionsoff
%  \usepackage[nomarkers]{endfloat}
% \let\MYoriglatexcaption\caption
% \renewcommand{\caption}[2][\relax]{\MYoriglatexcaption[#2]{#2}}
%\fi
% endfloat.sty was written by James Darrell McCauley, Jeff Goldberg and
% Axel Sommerfeldt. This package may be useful when used in conjunction with
% IEEEtran.cls'  captionsoff option. Some IEEE journals/societies require that
% submissions have lists of figures/tables at the end of the paper and that
% figures/tables without any captions are placed on a page by themselves at
% the end of the document. If needed, the draftcls IEEEtran class option or
% \CLASSINPUTbaselinestretch interface can be used to increase the line
% spacing as well. Be sure and use the nomarkers option of endfloat to
% prevent endfloat from "marking" where the figures would have been placed
% in the text. The two hack lines of code above are a slight modification of
% that suggested by in the endfloat docs (section 8.4.1) to ensure that
% the full captions always appear in the list of figures/tables - even if
% the user used the short optional argument of \caption[]{}.
% IEEE papers do not typically make use of \caption[]'s optional argument,
% so this should not be an issue. A similar trick can be used to disable
% captions of packages such as subfig.sty that lack options to turn off
% the subcaptions:
% For subfig.sty:
% \let\MYorigsubfloat\subfloat
% \renewcommand{\subfloat}[2][\relax]{\MYorigsubfloat[]{#2}}
% However, the above trick will not work if both optional arguments of
% the \subfloat command are used. Furthermore, there needs to be a
% description of each subfigure *somewhere* and endfloat does not add
% subfigure captions to its list of figures. Thus, the best approach is to
% avoid the use of subfigure captions (many IEEE journals avoid them anyway)
% and instead reference/explain all the subfigures within the main caption.
% The latest version of endfloat.sty and its documentation can obtained at:
% http://www.ctan.org/pkg/endfloat
%
% The IEEEtran \ifCLASSOPTIONcaptionsoff conditional can also be used
% later in the document, say, to conditionally put the References on a
% page by themselves.




% *** PDF, URL AND HYPERLINK PACKAGES ***
%
%\usepackage{url}
% url.sty was written by Donald Arseneau. It provides better support for
% handling and breaking URLs. url.sty is already installed on most LaTeX
% systems. The latest version and documentation can be obtained at:
% http://www.ctan.org/pkg/url
% Basically, \url{my_url_here}.




% *** Do not adjust lengths that control margins, column widths, etc. ***
% *** Do not use packages that alter fonts (such as pslatex).         ***
% There should be no need to do such things with IEEEtran.cls V1.6 and later.
% (Unless specifically asked to do so by the journal or conference you plan
% to submit to, of course. )


% correct bad hyphenation here
\hyphenation{op-tical net-works semi-conduc-tor}


\begin{document}
%
% paper title
% Titles are generally capitalized except for words such as a, an, and, as,
% at, but, by, for, in, nor, of, on, or, the, to and up, which are usually
% not capitalized unless they are the first or last word of the title.
% Linebreaks \\ can be used within to get better formatting as desired.
% Do not put math or special symbols in the title.
\title{Richardson-Lucy Deblurring \\ with Depth Information}
%
%
% author names and IEEE memberships
% note positions of commas and nonbreaking spaces ( ~ ) LaTeX will not break
% a structure at a ~ so this keeps an author's name from being broken across
% two lines.
% use \thanks{} to gain access to the first footnote area
% a separate \thanks must be used for each paragraph as LaTeX2e's \thanks
% was not built to handle multiple paragraphs
%

\author{Xiaoxin Fang, Bin Sheng

\thanks{X. Fang is at Shanghai Jiao Tong Uinversig}}%

% note the % following the last \IEEEmembership and also \thanks -
% these prevent an unwanted space from occurring between the last author name
% and the end of the author line. i.e., if you had this:
%
% \author{....lastname \thanks{...} \thanks{...} }
%                     ^------------^------------^----Do not want these spaces!
%
% a space would be appended to the last name and could cause every name on that
% line to be shifted left slightly. This is one of those "LaTeX things". For
% instance, "\textbf{A} \textbf{B}" will typeset as "A B" not "AB". To get
% "AB" then you have to do: "\textbf{A}\textbf{B}"
% \thanks is no different in this regard, so shield the last } of each \thanks
% that ends a line with a % and do not let a space in before the next \thanks.
% Spaces after \IEEEmembership other than the last one are OK (and needed) as
% you are supposed to have spaces between the names. For what it is worth,
% this is a minor point as most people would not even notice if the said evil
% space somehow managed to creep in.



% The paper headers
%\markboth{Journal of \LaTeX\ Class Files,~Vol.~14, No.~8, August~2015}%
%{Shell \MakeLowercase{\textit{et al.}}: Bare Demo of IEEEtran.cls for %IEEE Journals}
% The only time the second header will appear is for the odd numbered pages
% after the title page when using the twoside option.
%
% *** Note that you probably will NOT want to include the author's ***
% *** name in the headers of peer review papers.                   ***
% You can use \ifCLASSOPTIONpeerreview for conditional compilation here if
% you desire.




% If you want to put a publisher's ID mark on the page you can do it like
% this:
%\IEEEpubid{0000--0000/00\$00.00~\copyright~2015 IEEE}
% Remember, if you use this you must call \IEEEpubidadjcol in the second
% column for its text to clear the IEEEpubid mark.



% use for special paper notices
%\IEEEspecialpapernotice{(Invited Paper)}




% make the title area
\maketitle

% As a general rule, do not put math, special symbols or citations
% in the abstract or keywords.
\begin{abstract}
Motion blurred images that are captured in real world always have spatially-varying point spread functions(PSFs) due to different positions and depth values. This paper solve this problem by treating the blurred images as an integration of a sequence of clear images. We apply depth images in the projective motion Richardson-Lucy(RL) algorithm to tackle the problem of depth-dependent PSFs. We proposed a new method based on patch-match to fill the empty holes in depth images. Loopy brief propagation is used to smooth the depth image results by regrading the images as Markov Random Field (MRF). Deblurring and depth-filling is done iteratively to refine the results. Our method can bye also applied on real captured images with motion estimation given. The experimental results show that our method can handle depth-variant motion blur as well as to refine the depth images.
\end{abstract}

% Note that keywords are not normally used for peerreview papers.
\begin{IEEEkeywords}
Deblur, depth-variant, Richardson-Lucy.
\end{IEEEkeywords}






% For peer review papers, you can put extra information on the cover
% page as needed:
% \ifCLASSOPTIONpeerreview
% \begin{center} \bfseries EDICS Category: 3-BBND \end{center}
% \fi
%
% For peerreview papers, this IEEEtran command inserts a page break and
% creates the second title. It will be ignored for other modes.
\IEEEpeerreviewmaketitle



\section{Introduction}
\IEEEPARstart{D}{igital} images may have motion blur when captured by cameras. The final image contains all the light information in the exposure time. Motion blur is caused by the relative motion between the camera and the objects being recorded. Motion deblurring focused on recover the clear image information with the blurred image given.

A simple assumption about motion blur is to regard the blurring image as the convolution result of the clear image and a spatial-invariant point spread information(PSF). Noise and be also added to imitate real world capturing process. The deconvolution problem can be solved by optimization or Inverse Fourier Transformation.

However, spatial-invariant PSF is almost impossible for real cases because two reasons. One is that motions such as zoom-in(zoom-out) or rotations will lead to different pixels have different motion path. Another reason is because PSF varying with different depth values of each point. Spatial-variant PSFs make deblurring problems much complicate. Although optimization method can be also used in solving such problem by assuming different PSFs for each pixel, method in this way is time-consuming.

We use the Projective Motion Blur Model introduced by [P. Tan] to treat the blur images as an integration of a sequence of clear images. Each frame of the image sequence is the obtained by applying one transformation on the initial clear frame. In the view of each pixel, this transformation is implemented by using the pixel's coordinate multiplied by a 4x4 matrix. Here we use 4x4 matrix rather than 3x3 used in [P. Tan] because our method also considered the depth information. Since the transformation is applied on each pixel, spatial-varying PSFs is perfectly achieved. Because our method is no longer based on convolution, deblurring is done based on Richardson-Lucy(RL) algorithm rather than deconvolution. We extend the RL algorithm and make it able to handle spatial-variant motion blur in 3D space.

One problem in taking depth into consideration is that most depth images captured by current devices contains empty holes. In our deblurring model, each frame of the latent clear image sequence should have its corresponding depth map, which is obtained by applying the same transformation on the init clear depth map and so may contain empty holes. We use path match to help filling these holes. Then brief propagation is used to help smooth the edges and noise in the intermediate results. Deblurring and depth-filling are performed iteratively to refine the clear images as well as the depth map.

The remainder of this paper is organized as follows: ... to be continued.

\section{Related Work}
There are two major categories of image deblurring: blind deblurring and non-blind deblurring, depending on whether the point spread function(PSF) is given. Since the PSFs can be uniform or non-uniform, image deblurring can be further categorized into spatial-invariant or spatial-variant problems. Most previous work are focused on the blurring models with spatial-invariant PSF.

Non-blind deblurring methods can restore latent images with blur images and the corresponding PSF given. Blurring images in these cases can be synthesised with the ground truth blurring kernel. Traditional algorithms such as Wiener filter[2]��Richardson-Lucy algorithm [3] all tried to minimized the difference between the convolution results of latent images and the blurred ones. These methods suffer from the problem of ring artifacts because of the loss of high frequency details in the blurred image. Also, noise will be amplified in the optimization and deconvolution process. To handle these problem, regulation terms are used to constrain the restored image. Tikhonov regulation used the square of pixel intensity and Gaussian regulation used the square of the image gradient. Chan and Wong [4] proposed to use the L1 norm of the image gradient as the regularization term, that is, the Laplacian term. Levin et al.[5] extended this method, and make the regularization term can be any power of the image gradient and call it as the Hyper-Laplacian term. Later [6] set threads and apply different terms for different gradient values.

Compared to blind deblurring, non-blind deblurring methods have a new process to estimate the blurring kernel directly from the blurred image.
Tradition method like [4] applied the method of Laplician regularization term on estimating PSF. [8] pointed out that this method may have the problem of making the deblurring results the same as the source blurred image, and so the PSF is just one point with value one in the center. So [9] proposed the approach to estimate the PSF according to the transparency of the blurred object edges. Recent method such as [6] and [10] combined the deblurring and kernel estimation together and perform these two process iteratively to obtain the final latent image. Fourier Transformation [11] or Iteratively Re-weighted Least Squares (IRLS)[12] can be used in solve these optimization problem. But these methods cannot handle images with spatial-variant PSFs.

Above methods used the blur image as the only image reference, so have no enough information about the latent image. The deblurring quality varies from case to case and it's not robust enough. To get more details about clear images, [13] used the high-noised clear image captured with short exposure time to guide the deblurring process. [14] took two images
with different resolution use low-resolution high-frequency image as guidance to restore the high-resolution blurred ones. [15] use three different camera models to improve image quality. However, these methods require complicate devices to get the source input.

Recently, [Bardsley et al. 2006] proposed the method to estimate spatially variant kernel. Image segmentation are used in [Levin 2006] to set different PSFs for different layers. But these methods cannot solve the problem that each pixel has one PSF. Besides, with the development of artificial intelligence, machine learning is also used in [17] and [18] to do deblurring. But learning methods require the training set and the testing set should have similar blurring kernel, otherwise the deblurring result is useless. 

Our work based on the projection motion model by [P. Tan] and so achieve pixel level PSFs. We also add depth values to help improve the deblurring results of real world scenes. Unlike [16] which use depth to segment the image into layers, our work directly apply depth value of each pixel to help deblur this pixel itself. Although our work is non-blind deblurring method based on RL algorithm[3], our method can be extends to real cases with kernel estimation under user assistance.



\section{Conclusion}
The conclusion goes here.





% if have a single appendix:
%\appendix[Proof of the Zonklar Equations]
% or
%\appendix  % for no appendix heading
% do not use \section anymore after \appendix, only \section*
% is possibly needed

% use appendices with more than one appendix
% then use \section to start each appendix
% you must declare a \section before using any
% \subsection or using \label (\appendices by itself
% starts a section numbered zero.)
%


\appendices
% you can choose not to have a title for an appendix
% if you want by leaving the argument blank
\section{}
Appendix two text goes here.


% use section* for acknowledgment
\section*{Acknowledgment}


The authors would like to thank...


% Can use something like this to put references on a page
% by themselves when using endfloat and the captionsoff option.
\ifCLASSOPTIONcaptionsoff
  \newpage
\fi



% trigger a \newpage just before the given reference
% number - used to balance the columns on the last page
% adjust value as needed - may need to be readjusted if
% the document is modified later
%\IEEEtriggeratref{8}
% The "triggered" command can be changed if desired:
%\IEEEtriggercmd{\enlargethispage{-5in}}

% references section

% can use a bibliography generated by BibTeX as a .bbl file
% BibTeX documentation can be easily obtained at:
% http://mirror.ctan.org/biblio/bibtex/contrib/doc/
% The IEEEtran BibTeX style support page is at:
% http://www.michaelshell.org/tex/ieeetran/bibtex/
%\bibliographystyle{IEEEtran}
% argument is your BibTeX string definitions and bibliography database(s)
%\bibliography{IEEEabrv,../bib/paper}
%
% <OR> manually copy in the resultant .bbl file
% set second argument of \begin to the number of references
% (used to reserve space for the reference number labels box)
\begin{thebibliography}{2}

\bibitem{IEEEhowto:kopka}
H.~Kopka and P.~W. Daly, \emph{A Guide to \LaTeX}, 3rd~ed.\hskip 1em plus
  0.5em minus 0.4em\relax Harlow, England: Addison-Wesley, 1999.

\bibitem{IEEEhowto:Tai Y W}
Tai Y W, Tan P, Brown M S. Richardson-Lucy deblurring for scenes under a projective motion path[J]. IEEE Transactions on Pattern Analysis and Machine Intelligence, 2011, 33(8): 1603-1618.
\end{thebibliography}

% biography section
%
% If you have an EPS/PDF photo (graphicx package needed) extra braces are
% needed around the contents of the optional argument to biography to prevent
% the LaTeX parser from getting confused when it sees the complicated
% \includegraphics command within an optional argument. (You could create
% your own custom macro containing the \includegraphics command to make things
% simpler here.)
%\begin{IEEEbiography}[{\includegraphics[width=1in,height=1.25in,clip,keepaspectratio]{mshell}}]{Michael Shell}
% or if you just want to reserve a space for a photo:

\begin{IEEEbiography}{Xiaoxin Fang}
Biography text here.
\end{IEEEbiography}

% if you will not have a photo at all:
\begin{IEEEbiographynophoto}{Bin Sheng}
Biography text here.
\end{IEEEbiographynophoto}



% You can push biographies down or up by placing
% a \vfill before or after them. The appropriate
% use of \vfill depends on what kind of text is
% on the last page and whether or not the columns
% are being equalized.

%\vfill

% Can be used to pull up biographies so that the bottom of the last one
% is flush with the other column.
%\enlargethispage{-5in}



% that's all folks
\end{document}


